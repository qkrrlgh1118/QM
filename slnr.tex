 \documentclass[11pt,oneside,reqno]{amsart}
\usepackage{geometry}
\geometry{verbose,tmargin=1in,bmargin=1in,lmargin=1in,rmargin=1in}
\usepackage{amstext}
\usepackage{amsthm}
\usepackage{amssymb}
\usepackage{esint}
\usepackage{color}


\makeatletter
%%%%%%%%%%%%%%%%%%%%%%%%%%%%%% Textclass specific LaTeX commands.
\numberwithin{equation}{section}
\numberwithin{figure}{section}
\theoremstyle{definition}
\newtheorem{thm}{\protect\theoremname}[section]
  \theoremstyle{definition}
  \newtheorem{fact}[thm]{\protect\factname}
  \theoremstyle{definition}
  \newtheorem{defn}[thm]{\protect\definitionname}
  \theoremstyle{definition}
  \newtheorem{lem}[thm]{\protect\lemmaname}
  \theoremstyle{remark}
  \newtheorem{rem}[thm]{\protect\remarkname}
  \theoremstyle{definition}
  \newtheorem{cor}[thm]{\protect\corollaryname}
  \theoremstyle{definition}
  \newtheorem*{thm*}{\protect\theoremname}
  \theoremstyle{definition}
  \newtheorem{prop}[thm]{\protect\propositionname}

%%%%%%%%%%%%%%%%%%%%%%%%%%%%%% User specified LaTeX commands.


\usepackage[latin9]{inputenc}
\usepackage{geometry}
\setcounter{secnumdepth}{3}
\setcounter{tocdepth}{3}
\usepackage{amsmath}
\usepackage{amsthm}




%%%%%%%%%%%%%%%%%%%%%%%%%%%%%% Textclass specific LaTeX commands.
\numberwithin{equation}{section}
\numberwithin{figure}{section}
\usepackage{enumitem}		% customizable list environments
\newlength{\lyxlabelwidth}      % auxiliary length 
 \let\footnote=\endnote
\@ifundefined{lettrine}{\usepackage{lettrine}}{}
%%%%%%%%%%%%%%%%%%%%%%%%%%%%%% 

%%%%%%%%%%%%%%%%%%%%%%%%%%%%%% Theorem enviroment

%\newtheorem{}{}{thmdefn}{Definition/Theorem}
%\newtheorem{obs}{}[section]
%\newtheorem*{thmdefn}{Definition/Theorem}
%\newtheorem{thm}{Theorem}[section] % first theorem in section 1 will be 1.1
\theoremstyle{definition}
\newtheorem{thmx}{Theorem}
\renewcommand{\thethmx}{\Alph{thmx}} % "letter-numbered" theorems
%\newtheorem{obs}{Observation}[section]
%%%%%%%%%%%%%%%%%%%%%%%%%%%%%% User specified LaTeX commands.
\def\a{\alpha}
\def\b{\beta}
\def\d{\delta}
\def\s{\sigma}
\def\k{\kappa}
\def\t{\tau}
\def\SI{\Sigma}
\def\G{\Gamma}
\def\g{\gamma}
\def\l{\lambda}
\def\L{\Lambda}
\def\w{\omega}
\def\dd{\mathrm{d}}


\def\D{\mathbb{D}}
\def\R{\mathbb{R}}
\def\C{\mathbb{C}}
\def\H{\mathbb{H}}
\def\N{\mathbb{N}}
\def\Z{\mathbb{Z}}




\def\pa{\partial}
\def\ep{\varepsilon}
\def\vphi{\varphi}
\def\un{\underline}
\def\bb{\mathbb}
\def\cal{\mathcal}
\def\scr{\mathscr}


          

\def\a{\alpha}
\def\b{\beta}
\def\s{\sigma}
\def\Si{\Sigma}
\def\A{\mathcal{A}}
\def\G{\Gamma}
\def\g{\gamma}
\def\l{\lambda}
\def\L{\mathcal{L}}

\def\D{\mathbb{D}}
\def\P{\mathsf{P}}
\def\R{\mathbb{R}}
\def\C{\mathbb{C}}
\def\CC{\mathcal{C}}
\def\H{\mathbb{H}}
\def\H{\mathbb{H}}
\def\I{\mathrm{I}}
\def\J{\mathrm{J}}
\def\K{\mathrm{K}}
\def\W{\mathcal{W}}
\def\PR{\P(\R^d)}
\def\M{\mathcal{M}}
\def\Md{\mathsf{M}_d(\R)}
\def\vps{\vphi^s}

\def\w{\wedge}
\def\pa{\partial}
\def\ep{\varepsilon}
\def\blp{B_{\text{loc}}(p)}
\def\hol{H\"older }
\def\fb{fiber-bunched }
\def\mangle{\measuredangle}


\def\lc{\left\lceil}   
\def\rc{\right\rceil}
\def\t{\intercal}

\newcommand{\Wloc}{\mathcal{W}_{\text{loc}}}
\newcommand{\BV}{Bonatti-Viana}
\newcommand{\tilu}{\tilde{u}}
\newcommand{\tilv}{\tilde{v}}
\newcommand{\tilw}{\tilde{w}}
\newcommand{\edit}{\textcolor{red}{EDIT}}
\newcommand{\Sig}{\Sigma_T}
\newcommand{\slr}{\text{SL}_d(\R)}
\newcommand{\glr}{\text{GL}_d(\R)}

\makeatother
  \providecommand{\corollaryname}{Corollary}
  \providecommand{\definitionname}{Definition}
  \providecommand{\factname}{Fact}
  \providecommand{\lemmaname}{Lemma}
  \providecommand{\propositionname}{Proposition}
  \providecommand{\remarkname}{Remark}
  \providecommand{\theoremname}{Theorem}
\providecommand{\theoremname}{Theorem}

\begin{document}

\section{Setting}
\subsection{General set up}
Let $(\Sig,f)$ be a mixing subshift of finite type where $T$ is an primitive irreducible adjacency matrix with entries in $\{0,1\}$. Fix $\theta \in (0,1)$ and we endow $\Sig$ with a metric $d$ such that for $x = (x_i)_i,y = (y_i)_i \in \Sig$, we have
$$d(x,y)  = \theta^k,$$ 
where $k$ is the largest integer such that $x_i = y_i$ for all $|i| \leq k$. Let $\L$ be the collection of all admissible words, and for each $n \in \N$, let $\L(n) \subset \L$ be the set of admissible words of length $n$. For $\I = [i_0\ldots i_{n-1}] \in \L(n)$, we define the associated cylinder (also denoted by $\I$) as 
$$\I:=\{x \in \Sig \colon x_j = i_j \text{ for all } 0 \leq j \leq n-1\}$$ 

We define the \textit{local stable set} $\Wloc^s(x)$ \textit{of $x \in \Sig$} as 
$$\Wloc^s(x):=\{y \in \Sig \colon x_i = y_i \text{ for all } i \geq 0\}.$$
We extend the definition to define the \textit{stable set} $\W^s(x)$ \textit{of} $x$ as the set of $y \in \Sig$ such that there exists $n \geq 0$ with $f^ny \in \Wloc^s(f^nx)$. We define the \textit{(local) unstable set $\W^u$ of $f$} as the (local) stable set of $f^{-1}$.

For any $x,y\in \Sig$ with $x_0 = y_0$, we define $[x,y]:=\Wloc^s(x) \cap \Wloc^u(y)$. From the definition, $[x,y]$ is the unique point in the local neighborhood of $x$ and $y$ which shadows the orbit of $x$ in the future and the orbit of $y$ in the past.

\subsection{Fiber-bunched cocycles}
Let $A$ be an $\alpha$-\hol $\slr$-valued function on $\Sigma$: there exists $C>0$ such that
$$\|A(x)-A(y)\|  \leq Cd(x,y)^\alpha,$$
where $\|\cdot \|$ is the standard operator norm. 

From $A$, we can define the associated cocycle $\A \colon \Sig \times \mathbb{Z} \to \slr$ by $\A(x,0) = Id$ and for all $n \in \N$,
$$\A^n(x):=\A(x,n) = A(f^{n-1}x)\ldots A(fx)A(x) \text{ and }\A^{-n}(x) := \A(x,-n) = \big(\A^n(f^{-n}x)\big)^{-1}.$$
Throughout the article, we will assume that the cocycle $\A$ is \textit{fiber-bunched}: for all $x\in \Sig$,
$$\|A(x)\|\|A(x)^{-1}\| \theta^\alpha <1.$$ 
Equivalently, the fiber-bunching condition says that for every $x \in \Sig$, the amount $A(x)$ expands/contracts the fiber is necessarily smaller than the expansion/contraction of $f$. 


The \hol continuity and the fiber-bunching together ensure the convergence of the \textit{local stable/unstable holonomies} $H^{s/u}_{x,y}$: for any $y \in \Wloc^{s/u}(x)$, 
\begin{equation}\label{eq: hol}
H^s_{x,y} :=\lim\limits_{n \to \infty} \A^n(y)^{-1}\A^n(x) ~\text{ and }~H^u_{x,y}:= \lim\limits_{n \to -\infty} \A^n(y)^{-1}\A^n(x).
\end{equation}

The following proposition lists the properties of the holonomies defined as in $\eqref{eq: hol}$.
\begin{prop}\label{prop:  hol rel} There exists $C>0$ such that for any $y,z \in \Wloc^s(x)$ and for any $n \geq 0$, 
\begin{enumerate}
\item $H^s_{x,x} = Id$ and $H^s_{x,y}\circ H^s_{y,z} = H^s_{x,z}$,
\item $\A^n(x) = H^s_{f^ny,f^nx} \circ \A^n(y) \circ H^s_{x,y}$,
\item $\|H^{s}_{x,y} - Id\| < Cd(x,y)^{\alpha}$.
\end{enumerate}
The similar relations hold for the unstable holonomies.
\end{prop}

\begin{rem}
It is worth noting a special family of cocycles trivially admitting the holonomies: when $A(x)$ depends only on the zero-th entry $x_0$ of $x$ (ie, \textit{locally constant}), then both $H^s$ and $H^u$ from \eqref{eq: hol} trivially converge to $Id$ and satisfy properties from Proposition \ref{prop:  hol rel}. By decreasing $\theta$ if necessary, a locally constant cocycle is fiber-bunched.
\end{rem}

Using (2) of Proposition \ref{prop:  hol rel}, we can extend the definition of local stable holonomies $H^s_{x,y}$ for $y \in \W^s(x)$ not ncessarily in the local stable set of $x$ by
$$H^s_{x,y}:= \A^n(y)^{-1} H^s_{f^nx,f^ny}\A^n(x),$$
where $n \in \N$ is large enough so that $f^ny \in \Wloc^s{f^nx}$. We can likewise define unstable holonomies.  

We now formulate the main assumptions of the theorem. Consider any periodic point $p$ and any of its homoclinic point $z \in \W^s(p )\cap \W^u(p)$. We define the \textit{holonomy loop} $\psi_p^z$ as the composition of the unstable holonomy from $p$ to $z$ and the stable holonomy from $z$ to $p$:
$$\psi_p^z  := H^s_{z,p}\circ H^u_{p,z}.$$
\begin{defn}
Following Bonatti-Viana, we say the cocycle $\A$ is \textit{typical} if it is \hol and \fb, and satisfies two extra assumptions:
\begin{enumerate}
\item[A.] there exists a periodic point $p$ such that $P:=\A^r(p)$ has simple eigenvalues where $r$ is the period of $p$,  
\item[B.] there exists a homoclinic point $z$ of $p$ such that $\psi_p^z$ twists the eigenspaces of $P$ in general position: denoting the eigenvectors of $P$ by $\{v_1,\ldots,v_n\}$, for any $1 \leq i \leq n$, $\psi_p^z(v_i)$ does not lie in the hyperplane spanned by all $v_j$ with $j \neq i$.
\end{enumerate}
\end{defn}
\begin{rem}
A few comments regarding the assumptions are in order. First, we may assume $p$ is a fixed point by replacing $f$ by the suitable power of $f$. Second, by replacing $z$ by its suitable backward iterate, we may assume $z \in \Wloc^u(p)$. We can then fix $l \in \N$ such that $f^lz \in \Wloc^s(p)$. From Proposition \ref{prop:  hol rel}, we can rewrite $\psi_p^z$ as
\begin{equation}\label{eq: psi_p^z}
\psi_p^z=P^{-l} \circ H^s_{f^lz,p}\circ\A^l(z) \circ H^u_{p,z}.
\end{equation}


\end{rem}

Such assumptions are first introduced by Bonatti-Viana as sufficient conditions to establish the simplicity of the Lyapunov exponents of ergodic $f$-invariant measures with continuous local product structure.

\subsection{Thermodynamics formalism}
We define the \textit{singular value function} $\vps \colon \Md \to \R$ with parameter $s\geq 0 $ as follows:
$$\vps(M) = \begin{cases} 
      \alpha_1(M)\ldots\alpha_{\lfloor s \rfloor}(M)\alpha_{\lceil s \rceil}(M)^{\{s\}} & 0\leq s \leq d ,\\
      |\det(M)|^{s/d} & d < s
   \end{cases}$$
where $\alpha_1(M) \geq \ldots \geq \alpha_d(M)$ are the singular values (the square roots of the eigenvalues of $M^*M$) of $M$. 
It is well known that $\vps$ is submultiplicative for all $s$ (See Falconer's famous paper from 88; lemma 2.1):
\begin{equation}\label{eq: submult}
\vps(MN) \leq \vps(M) \vps(N).
\end{equation}
Moreover, the function $(M,s) \mapsto \vps(M)$ is upper semi-continuous, and has discontinuity at the integer $s=k$ where the singular values jump $\alpha_{k-1}(M)>\alpha_k(M) = 0$. If $M$ takes value in $\glr$, then $\vps(M)$ is continuous in both $M$ and $s$.
The singular value function is widely used in dimension theory of self-affine sets (some refs).

When $\vps$ is applied to the cocycle $\A$, $\vps(\A^n(\cdot))$ for each $n \in \N$ form a submultiplicative sequence of functions on $\Sig$. In particular, the cocycle can be studied via thermodynamics formalism. 
In the following lemma, we show that $\vps$ is uniformly controlled within a cylinder for a \fb cocycle.
\begin{lem}[bounded distortion]\label{lem: bdd distortion} Let $\A$ be \hol and fiber-bunched.
Given $s \in [0,\infty)$, there exists $C>1$ such that for any $n \in \N$, any cylinder $\I \in \L(n)$, and any $x,y \in \I$, 
$$C^{-1} \leq \frac{\vps(\A^n(x))}{\vps(\A^n(y))} \leq C.$$ 
\end{lem}
\begin{proof}
From (3) of Proposition \ref{prop: hol rel}, we can fix $c>1$ such that the norm of any local holonomy is bounded above by $c$. In particular, $\vps$ evaluated at any local holonomy is uniformly bounded above by $c^s$. 
Then, setting $z:=[x,y]$ and using (2) of Proposition \ref{prop:  hol rel} as well as \eqref{eq: submult}, we have 
$$c^{-2s}\vps(\A^n(x)) \leq  \vps(\A^n(z)) =\vps( H^s_{f^nx,f^nz} \circ \A^n(x) \circ H^s_{z,x}) \leq c^{2s} \vps(\A^n(x)).$$
Using unstable holonomies instead, we similarly have $c^{-2} \leq \vps(\A^n(y))/\vps(A^n(z))\leq c^2$.
Then, the statement follows by setting $C = c^{4s}$.
\end{proof}
We can naturally consider $\vps$ as a function on the set of admissible words $\L$ in that
$$\vps(\I) := \sup\limits_{x \in \I} \vps(\A^n(x)).$$
The uniform control of $\vps$ within a cylinder from Lemma \ref{lem: bdd distortion} sufficient (for instance, in the definition of the Gibbs measure). 

We can associate the \textit{singular value pressure} to a cocycle $\A$:
\begin{equation}\label{eq: sing pressure}
\P_\A(\vps) = \lim\limits_{n \to \infty} \frac{1}{n}\log\Big( \sum\limits_{\I \in \L(n)} \sup\limits_{x \in \I} \vps(\A^n(x))\Big)=\lim\limits_{n \to \infty} \frac{1}{n}\log\Big( \sum\limits_{\I \in \L(n)}  \vps(\I)\Big).
\end{equation}
The existence of the limit is guaranteed from the submultiplicativity \eqref{eq: submult} of the singular value function. The usual (additive) variational principle (Walters) extends to the subadditive potentials (Cao-Huang-Feng): denoting the space of $f$-invariant probability measures by $\M(f)$, we have
\begin{equation}\label{eq: var prin}
\P_\A(\vps) = \sup\limits_{\mu \in \M(f)} \Big( h_\mu(f)+\lambda_\A(\vps,\mu)\Big),
\end{equation}
where $h_\mu(f)$ is the measure-theoretic entropy (Walters or maybe Sinai?) and $$\lambda_\A(\vps,\mu) = \lim\limits_{n \to \infty} \frac{1}{n} \int \log \vps(\A^n(x)) d\mu(x) = \inf\limits_{n  \to \infty}\frac{1}{n} \int \log \vps(\A^n(x)) d\mu(x),$$
whose limit is also guaranteed to exist from the submultiplicativity \eqref{eq: submult}. Unique zero (strict decrease) of $s$ affinity dimension, and continuity of and relation to study of self-affine sets and Lyapunov dimension and Hausdorff dimension.

Bounded distortion allows us to treat the subadditive sequence as a subadditive sequence on $\L$.
\begin{lem}
Let $\Psi$ be a subadditive sequence of functions with bounded distortion. Then, the equilibrium states of $\Psi$ are the same as the equilibrium states of
\end{lem}

The supremum from \eqref{eq: var prin} is always achieved (from expansivity of the shift? check Cao-Feng-Huang), and any $\mu \in \M(f)$ attaining the supremum is called an \textit{equilibrium state}. While,  unique equilibrium state sufficient condition which we describe below. Unique equilibrium state arises as Gibbs measure which has applications....


\begin{defn} The singular value function $\vps$ for a \fb cocycle is \textit{quasimultiplicative} if there exists $c>0$ and $k \in \N$ such that for any two cylinders $\I,\J \in\L$, there exists $\K = \K(\I,\J)\in \bigcup\limits_{i=1}^k\L(i)$ such that $\I\K\J \in \L$ and that
$$\vphi^s(\I\K\J) \geq c \vphi^s(\I)\vphi(\J).$$ 
\end{defn}
 
Comment on the quasimultiplicativity. Used in blah blah. Locally constant setting irreducibility gives QM.

The main reason why we want to establish quasimultiplicativity is the existence of the unique Gibbs measure: Feng establishes a sufficient condition for the existence of the unique, and hence ergodic, equilibrium state. 
\begin{prop}[Feng-Factor maps; existence of the unique equilibrium state]
Let $\Phi \colon \L \to [0,\infty)$ be a submultiplicative and subadditive 
Gibbs property
\end{prop}

Such Gibbs measure is the unique equilibrium state.


In which section, we show that the singular value function of a typical cocycle is quasimultiplicativce. Our main work.
\begin{prop}
The singular value function $\vps$ of a typical cocycle $\A$ is quasimultiplicative for all $s \in [0,\infty)$.
\end{prop} 
 


\subsection{Multilinear Algebra}
We will make use of exterior algebra in studying 


\section{Results (probably move upward; embed in the introduction)}



\subsection{Singular value potential}
\begin{thm}
For a typical cocycle, the singular value potential $\vps$ is quasimultiplicative for all $s \in (0,d)$.
\end{thm}

%\begin{thm}
%If $P$ is semisimple, and $P$ and $\psi_p^z$ are irreducible, then the cocycle is quasimultiplicative.
%\end{thm}

\begin{thm}
The set of all possible pointwise Lyapunov exponent is an interval. 
\end{thm}
Maybe extend this to multi dimension?

\begin{thm}
Multifractal analysis.
\end{thm}
Also, possible to extend to multi dimension?

\begin{rem}
Bonatti-Viana's condition clearly satisfies the assumption. It is generic.
\end{rem}

%\begin{rem}
%Comment on the semisimplicity assumption on $P$.
%\end{rem}


\newpage
\section{QM}
In this section, we prove quasimultiplicativity of $\vps$ for $s\in \N$ by using $p$ as the reference point where we compare the directions responsible for the norm growth. We will first illustrate the idea when $\A$ is a locally constant cocycle. We will not distinguish between a vector and its corresponding line in the projective space when there is no confusion.


For any $\R$-vector space $V$ with a norm $\|\cdot\|$ and any $A \in \text{End}(V)$, we define $v(A)$ and $u(A)$ as the most expanding direction of $A$ and $A^*$, respectively. 

The following notations and constants will be used in common in what follows. The eigenvalues of $P$ are denoted by $\lambda_1 > \ldots > \lambda_d$ with corresponding eigenvectors $v_1,\ldots,v_d$. The adjoint $R:=P^*$ has identical eigenvalues as $P$ with corresponding eigenvectors $w_i$'s where $w_i = \text{span}(v_1,\ldots,v_{i-1},v_{i+1},v_{n})^\perp$. Also, the adjoint $(\psi_p^z)^*$ of the holonomy loop twists $w_i$'s into general position.

We denote the cone around $v$ of size $\ep$ by $\CC(v,\ep):=\{w \colon \mangle (v,w ) < \ep\}$. Also, $\ep>0$ is a small constant such that $\psi_p^zv \not \in \CC(v_j,2\ep)$ for all $1 \leq j \leq d$ whenever $v \in \CC(v_i,\ep)$ for some $1 \leq i \leq d$. 

\subsection{Locally constant cocycle}

Given any two cylinders $\I,\J \in \L$, we will bring them close $p$ in finite iterates, and compare their directions which capture the behavior of $\vps$ to the eigendirections of $P$ which serve as the reference direction.

For simplicity, we will describe the method for $\vphi^1$. For other $s \in \N$, it simply amounts to replacing $\A$ by its induced map $\A^{\wedge s}$ and $\R^d$ to $(\R^d)^{\wedge s}$. 

Since local holonomies are all equal to the identity for a locally constant cocycle, in view of \eqref{eq: psi_p^z} the holonomy loop $\psi_p^z$ is simply equal to $$\psi_p^z = P^{-l}\A^l(z).$$


Since the adjacency matrix $T$ is primitive, for any $\I\in \L$, there exists $\K_{\I} \in \L$ whose length is bounded above independent of $\I$ and the end of $\tilde{\I} = \I\K_{\I}$ is at the local neighborhood of $p$. Then we have two cases for $u(\A(\I))$:
\begin{enumerate}
\item  $u(\A(\tilde{\I})) \not\in \CC(v_i ,\ep)$ in for any $i$, and hence, a bounded number (depending on $\ep$ only) of $P$ maps $u(\A(\tilde{\I}))$ into  $\CC(v_i,\ep)$ for some $i$,
\item $u(\A(\tilde{\I})) \in \CC(v_i ,\ep)$ already for some $i$.
\end{enumerate}
Either case, we apply $\A^l(z) = P^l\psi_p^z$ unless $i = 1$. Since $\psi_p^z$ maps each $v_i$ into general position with respect to all $v_i$'s, $P^l$ following $\psi_p^z$ will map $u(\A(\tilde{\I}))$ toward $v_1$. Applying more (yet bounded) $P$ if necessary, we can turn $u(\A(\tilde{\I}))$ as close to $v_1$ as we want. The only thing that matters is that angle $\A(\K_{\tilde{\I}})u(\A(\tilde{\I}))$ can be made as close to $v_1$ as needed with the length of $\K_\I\K_{\tilde{\I}}$ bounded above (by $k_1 \in \N$) independent of $\I$.

The reason we wanted to map $u(\A(\tilde{\I}))$ close to $v_1$ under $\A(\K_{\tilde{\I}})$ is because the angle $\mangle (v_1,w_1)$ is necessarily bounded away from $\pi/2$. Hence, when we turn $v(\A(\tilde{\J}))$ close to $w_1$ by applying the analogous argument to the adjoint (ie, $\A(\K_{\tilde{\J}})^*v(\A(\tilde{\J}))$ is close to $w_1$ with $|\K_{\tilde{\J}}\K_\J| \leq k_2$ for some $k_2 \in \N$), then the angle $\mangle\big(\A(\K_{\tilde{\I}})u(\A(\tilde{I})),\A(\K_{\tilde{\J}})^* v(\A(\tilde{\J}))\big)$ is necessarily uniformly  bounded away (independent of $\I$ and $\J$) from $\pi/2$. Then $\K = \K_\I\K_{\tilde{\I}}\K_{\tilde{\J}}\K_\J$ is the required connecting cylinder for the quasimultiplicativity whose length is bounded above by $k_1+k_2$: denoting $\beta <\pi/2$ to be an upper bound on the angle, $M :=\max\limits_{x \in \Sig}\|A(x)\|$ and $m:=\min\limits_{x \in \Sig}m(A(x))$, we have 
\begin{align*}
\cos \mangle\big(v(\A(\tilde{\J})), \A(\K_{\tilde{\J}})\A(\K_{\tilde{\I}})u(\A(\tilde{\I}))\big) M^{k_1+k_2}
&\geq\big\langle v(\A(\tilde{\J})), \A(\K_{\tilde{\J}})\A(\K_{\tilde{\I}})u(\A(\tilde{\I})) \big\rangle,\\
&= \big\langle \A(\K_{\tilde{\J}})^*v(\A(\tilde{\J})), \A(\K_{\tilde{\I}})u(\A(\tilde{\I})) \big\rangle,\\
&\geq \cos(\beta)\|\A(\K_{\tilde{\J}})^*v(\A(\tilde{\J}))\|\|\A(\K_{\tilde{\I}})u(\A(\tilde{\I}))\|,\\
&\geq \cos(\beta)m(\A(\K_{\tilde{\J}}))m(\A(\K_{\tilde{\I}})),\\
&\geq \cos(\beta) m^{k_1+k_2},
\end{align*}
and hence $\cos \mangle\big(v(\A(\tilde{\J})), \A(\K_{\tilde{\J}})\A(\K_{\tilde{\I}})u(\A(\tilde{\I}))\big) \geq \cos(\beta)(m/M)^{k_1+k_2}$.
It then follow that
\begin{align*}
\|\A(\I\K\J)\| &\geq \|\A({\tilde{\J}})\A(\K_{\tilde{\J}})\A(\K_{\tilde{\I}})\A({\tilde{\I}})v(\A(\tilde{\I}))\|,\\
&\geq \|\A(\tilde{\I})\|\|\A({\tilde{\J}})\A(\K_{\tilde{\J}})\A(\K_{\tilde{\I}}) u(\A(\tilde{\I}))\|,\\
&\geq \cos(\beta)(m/M)^{k_1+k_2}\|\A(\tilde{\I})\|\|\A(\tilde{\J})\|,\\
&\geq \cos(\beta)(m/M)^{k_1+k_2}m^{k_1+k_2}\|\A(\I)\|\|\A(\J)\|.
\end{align*}

\subsection{Not necessarily constant cocycle}
Moving away from a locally constant cocycle to an arbitrarily typical cocycle, the argument is more-or-less the same, but we have to address a couple of issues:
\begin{enumerate}
\item Connect/shadow orbits; via taking bracket $[\cdot, \cdot]$
\item Ensure that the quasimultiplicativity of $\vps$ remains to hold (by decreasing the constant if necessary) after taking the bracket. 
\item Identify the fibers along $\Wloc^s(p)$ and $\Wloc^u(p)$ by local holonomies. 
If $w$ is not on those two (stable or unstable) sets of $p$, then show that the loop $R_w$ is close to the identity. Here, $R_w$ for any $w$ in the local neighborhood is defined as the clockwise composition of the holonomies around the rectangle having $p$ and $w$ as opposite vertices:
$$R_w : = H^u_{[w,p],p}\circ H^s_{w,[w,p]}\circ H^u_{[p,w],w}\circ H^s_{p,[p,w]}.$$


\end{enumerate}












\newpage
$\I$ with length $a$ and $\J$ with length $b$. Denote $x,y,z,$ to label points in the local neighborhood of $p$. Related points with be indicated with tildes and hats.

Recall that $v_i$ is the eigenvector corresponding to the $i$-th eigenvalue $\lambda_i$ of $P$. We define $$\ep_0=\min\limits_{1\leq i,k\leq d} \mangle(v_i,\psi_p^zv_j)$$ as the minimum angle of the eigendirections by $\psi_p^z$. We fix $0<\ep_1 \ll \ep_0$ such that any $v \in \PR$ with $\mangle(v,v_i)$ for some $i$ satisfies $\mangle(\psi_p^zv,v_j) >\ep_0/2$ for any $1 \leq j \leq d$. Exponential convergence. Any $v \in v_i\setminus v_{i-1}$ converges exponentially to $v_{i}$ with rate $\lambda_i/\lambda_{i-1}$.
Can be turned to $v_1$ with finite number of iterates. Bring close to $p$. If already 

Using holonomies to identity the fibers, 

\begin{lem}
Given $\ep_1>0$ as above, there exists $c,k>0$ such that for any $\I \in \L(n_0)$, there exists $v \in S^{d-1}$ such that $A^n(x)v$ is $\ep$-close to $v_1$ and $\|A^n(x)v\| \geq c\|A^n(x)\|$. 
\end{lem}
\begin{proof} Since $(\Sig,f)$ is the mixing subshift of finite type, there exists $k_1$ depending only on the mixing rate of $f$ such that any two $\theta$-ball. Let $x \in \I$ such that $f^{a+k} x\in\Wloc^s(p)$. 

Consider $v(\A^{a+k}(x))$. There exists $m \in \N$ such that $v :=v(\A^{a+k+m}(x))$ is $\ep_1$ close to some $v_i$. If $i =1$, we are done. If not, twist and apply $P$.
\end{proof}

For any $w$ in the local neighborhood of $p$, we define $R_w$ as the clockwise composition of the holonomies around the rectangle having $p$ and $w$ as opposite vertices:
$$R_w : = H^u_{[w,p],p}\circ H^s_{w,[w,p]}\circ H^u_{[p,w],w}\circ H^s_{p,[p,w]}.$$
From ---, we can bound the norm of $R_w$ as the distance $d(p,w)$.

\begin{lem}[shadowing]
Given $\ep>0$, there exists $c,\d>0$ such that the following holds: 
suppose $x$ and $y$ are $\d$-close and $v$ and $w$ good. then 
$$\vphi^s(A^{n+m}z) \geq c \vphi^s(A^n(z))\vphi^s(A^m(f^nz))$$
\end{lem}


\begin{lem}[Angle may change but still captures the growth well] Given $\ep$, there exists $\delta,c>0$ such that the following holds: for any $x,y \in \Sigma$ and $n,m \in \N$ such that $d(f^nx,y)<\delta$ and $\mangle u(A^n(x)),v(A^m(y)^\perp)>\epsilon$, then there exists $z$ such that 
$$\vphi^s(A^{n+m}(z)) \geq c \vphi^s(A^n(x))\vphi^s(A^m(y)).$$ 
\end{lem}
\begin{proof}
Using the holonomy, $H^u_{f^nx,f^nz}u(A^n(x))$ captures the norm of $A^n(z)$ and is close to $u(A^n(x))$.
\end{proof}


If $\psi$ moves level, then we can do. 


\subsection{Adjoint action}


\section{To do}
\begin{itemize}
\item Given $\I,\J$, find $\K = \K(\I,\J)$ such that $\K$ works for all $s \in \N$ simultaneously. 
\item $\varphi^k,\varphi^{k+1}$ are quasimultiplicative for $k \in \N$, then $\vps$ is qm for all $s \in [k,k+1]$.
Write $s = ka+(k+1)(1-a)$ for $a \in (0,1)$. Then take $c_s = c_k^{a}c_{k+1}^{1-a}$. This works because we are choosing the common $\K$ for $k$ and $k+1$.
\item  
Simultaneous (multidimensional) fractal analysis. Carefully chosen details. But write up quasimultiplicativity first. Also, make sure to add explanation when passing to exterior algebra, everything works fine (similar to $\|\cdot\|$ case). $u,v$ directions are well-defined, and any wedge product converges to $v_1 \wedge \ldots \wedge v_k$ under the multiple application of $P$. Define angle in exterior algebra (by induced inner product or on Grassmann?).

For simultaneous fractal analysis, $N_k:=\L(k)$. The bigger $N_k$ the better.  But for reasonable $t_1,t_2,\ldots.$ Care needs to be taken for extreme combination of $t_i's$. 

\item $k$-typical imply ---. Ie, define typical as $k$-typical for all $k$. Have statement like $k-$typical then statement holds upto $k$-dimensional vector, 
\item Clear write up, multilinear algebra, quantifiers for locally constant case -- set up constants, definition for $k$-typical. Check detail for (describe how) bounded distortion enables us to think of $\A$ as locally constant cocycle in view of thermodynamics formalism. State theorems correctly. Add introduction. Add bib. Check Pesin et. all for the unique equilibrium parts and chapter 7.
\item Fiber-bunching is sufficient to do thermodynamics formalism; ie, equilibrium state.
\item One-sided or two-sided
\end{itemize}

\end{document}
